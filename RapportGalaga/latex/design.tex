\section{Design}

For at gøre et spil underholdende, skal det helst være muligt at tabe (og ligeså at vinde). Af denne årsag har vi lavet forskellige states, der stopper spillet og viser en menu, som der viser "You win" eller "You lost" alt efter omstændighederne. Spillet vindes ved, at alle fjender bliver skudt.
Da der ikke er blevet implementeret nogen måde for spilleren at dø (altså ved eksempelvis at støde ind i spilleren), så er den condition, at spillet tabes så snart en modstander rammer bunden af skærmen, blevet sat. Mere præcist benyttes, at man kender til fjendernes position, idet de er Entities med en position, der kan tjekkes om hvorvidt nogen af dem er nået til bunden af skærmen.
En anden state er, at spillet til et vilkårligt tidspunnkt kan pauses ved at trykke escape og kan fortsættes ved at trykke "continue". Dette har vist sig oplagt at implementere vha. en event, da vi allerede har en event-processor til at tjekke, hvilke keys, som bliver trykket, og i tilfælde af at den pågældende knap er "escape", da bliver der skiftet til en pause-state.
For at skabe blot en anelse mere spænding i spillet, er der blevet implementeret bevægelse samt formationer hos modstanderne. Formation er skabt ved at instantiere modstanderne de rigtige steder og så dernæst sørge for, at de bevæger sig uniformt for at opretholde formationen. I den bevægende formation ZigZagDown, foregår bevægelsen eksempelvist ved at rykke på fjendernes position som en sinusbølge, således at de får en flydende bølge-agtig bevægelse. For at undgå, at denne bevægelse resulterer i, at fjenderne bevæger sig udenfor skærmen, har vi siden forhenværende aflevering fjernet nogen fjender således, at udstrækningen af bevægelsen ikke er større end hvad skærmen tillader. 
