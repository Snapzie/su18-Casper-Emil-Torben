\section{Evaluering}
Vi har til dette projekt lavet tests løbende for at sikre os at koden har den intenderede funktion. Disse tests består af unit tests og integration tests. Vores unit test består i at teste at File Reader kan læse en fil og konvertere denne til et objekt af typen Level. Dette testes på to forskellige filer der har forskelligt antal platforme, kunder og forhindringer, i og med denne passer uden fejl regner vi med at fil til Level objekt fungerer som den skal.\\
Vores integration test består i at teste hele kommunikationskæden fra LevelLoader til LevelsKeeper. Testen starter med at kalde LevelLoader og få denne til at loade alle levels og derefter tjekker vi LevelsKeeper for at se om denne indeholder de forventede Level-objekter. Testen indeholder på den måde både LevelLoader, FileReader, Level klassen, og LevelsKeeper. Da denne test også passer kan vi regne med at hele denne kæde fungerer som den skal.\\
I sidste aflevering havde vi et problem som gjorde det umuligt at teste vores implementationer hvori der indgik Images fra DIKU-arcade. I denne aflevering har vi fundet en løsning til nogle af disse problemer, vi har nemlig lavet en Mock up af IBaseImage. Det der gjorde vi ikke kunne teste implementationen sidst var at Image havde en Render-method som krævede at der var et Game-window åbent, hvilket vi gerne vil undgå til vores tests. Vi har derfor lavet metoden CreateEntity om så denne tager et objekt der implementerer IBaseImage og da bruger den dette til at sætte på den entity den returnerer. Det betyder at vi i vores tests kan give den et MockUpImage, som implementerer IBaseImage, som argument, mens vi i selve spillet giver den et rigtigt Image. Dette betyder også at klassen EntityCreator ikke længere er dirrekte afhængig af Image. Vi har på den måde lavet vores kode mere abstrakt. Det betyder at hvis vi eksempelvis ville have nogle blokke der brugte ImageStrides i stedet for blot images, kunne vi gøre dette uden at skulle ændre i klassen EntityCreator.
