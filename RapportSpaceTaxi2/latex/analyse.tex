\section{Analyse}
Vi har indelt problemet i forskellige concepter, der hver har deres eget ansvar, resultatet kan ses herunder.\\
\begin{table}[!h]
\centering
\begin{tabular}{|l|l|}
\hline
\multicolumn{1}{|c|}{\textbf{Responsibility description}}            & \multicolumn{1}{c|}{\textbf{Concept name}} \\ \hline
Læser fil og returnerer Level objekt med info herfra        & File reader (FR)                       \\ \hline
Class til at beskrive Level objekter                        & Level class (LCl)                       \\ \hline
Singleton der skal indeholde en dictionary af Levels        & Levels keeper (LK)                    \\ \hline
Sørger for alle levels loades før start på spil             & Level loader (LL)                     \\ \hline
Laver en entity container til level baseret på Level objekt & Level creater (LCr)                     \\ \hline
Returnerer en entity som repræsenterer blok i spillet       & Entity creater (EC)                    \\ \hline
Sørger for at skifte mellem vores states & StateMachine \\ \hline
Vores game states & GameState\\ \hline
Sørger for at der kan subscribes til events & SpaceBus\\ \hline
\end{tabular}
\caption{Dele af vores løsning inddelt i responsibilities og koncepter}
\label{responsibilities}
\end{table}\\
Vi har opdelt koncepterne efter koncepttyperne Know og Do, så hvert koncept har netop én type. Vores Do koncepter har hver ansvar for én opgave, eksempelvis file reader der kun skal lave et level ud fra informationen fra en fil. Vi har 9 koncepter hvoraf 8 er Do- og 1 er know-koncepter.\\
Vores GameState koncept omhandler flere klasser som er ens i struktur men variere i funktionalitet således at vi for eksempel kan pause og tabe.\\
I tabel \ref{associations} herunder ses det hvordan de forskellige koncepter skal kommunikere med hinanden.
\begin{table}[!h]
\centering
\begin{tabular}{|l|l|l|}
\hline
\textbf{Concept pair}           & \textbf{Association description}      & \textbf{Association name} \\ \hline
Level Loader og file reader     & Giver filnavn og får level tilbage    & Hent data                 \\ \hline
File reader og level class      & Laver Level objekt                    & Genererer                 \\ \hline
Level loader og levels keeper   & Levels keeper gemmer Level fra loader & Gem data                  \\ \hline
Level creator og levels keeper  & Får Level objekt fra Keeper           & Hent data                 \\ \hline
Level creator og entity creator & Får entity med givne parameter        & Genererer                 \\ \hline
GameStates til StateMachine & Anmoder om at skifte til en ny state & request \\ \hline
StateMachine til GameStates & Giver inputevents videre & Kommunikerer \\\hline
SpaceBus og GameStates & GameStates registrerer events til SpaceBus & Kommunikerer\\\hline
SpaceBus og StateMachine & SpaceBus kommunikerer input events & Kommunikerer \\\hline
\end{tabular}
\caption{Associationer mellem forskellige concepter}
\label{associations}
\end{table}\\
Det ses at vi har besluttet at Level loader skal kalde en funktion i file reader for at denne returnerer et Level som Level loader gemmer i levels keeper. I starten af planlægningsprocessen var det egentlig planen at file reader skulle sende Level-objektet til levels keeper, men da vi fik et bedre overblik over de forskellige ansvarsområder kunne vi se at det gav bedre mening at gøre det som beskrevet ovenfor. På denne måde får vi nemlig en klarere adskillelse mellem Level loaders og file readers ansvar, i og med at file readers eneste job er at læse og parse filen, hvorefter level loader behandler Level-objektet.\\
I tabel \ref{attributes} ses det hvilke attributer vi tænker at hvert concept skal have.
\begin{table}[!h]
\centering
\begin{tabular}{|l|l|l|}
\hline
\textbf{Concept}               & \textbf{Attribute} & \textbf{Attribute description}                              \\ \hline
Level loader                   & Levels             & En liste af filnavnene på level layouts                     \\ \hline
File reader                    & Ingen attributer   &                                                             \\ \hline
\multirow{5}{*}{Level class}   & Level array        & 2D-struktur der indeholder level design                     \\ \cline{2-3}
                               & Name               & Navn på level                                               \\ \cline{2-3}
                               & Costommer          & Information om costommers i dette level                     \\ \cline{2-3}
                               & Platforms          & Info om hvad der er platforme                               \\ \cline{2-3}
                               & Decoder            & Gemmer sammenhængen mellem karakterer og billedenavne       \\ \hline
\multirow{2}{*}{Levels keeper} & Levels             & Contatiner til alle levels                                  \\ \cline{2-3}
                               & Index              & Tæller der holder styr på hvor mange levels der er I levels \\ \hline
Level creator                  & Levels             & Reference levels keepers                                    \\ \hline
Entity creator                 & Ingen attributer   &                                                             \\ \hline
StateMachine & ActiveState & Refference til den aktive state \\\hline
GameState & Instance & Refference til sig selv (Singleton)\\\hline
SpaceBus & Ingen attributer & \\\hline
\end{tabular}
\caption{Attributer for hvert koncept}
\label{attributes}
\end{table}\\
Det ses at det ikke er alle vores koncepter der har attributer, dette skyldes at det ikke er dem alle der skal gemme informationer. Eksempelvis skal file reader blot læse filen og generere objekter på baggrund af dette, den behøver altså ikke gemme informationen til senere brug. I modsætning til dette skal hvert Level objekt gemme informationen om den selv.\\
